\begin{flushleft}
	The following architecture constraints have 
	been selected by our client as being 
	suitable for the system.
\end{flushleft}

\begin{enumerate}
	
	\item \textbf{Development environment:} Linux
	\item \textbf{Development platform:} Java-Enterprise Edition (Java-EE) 
	\item \textbf{Architectural frameworks:} Java Server Faces (JSF)
	\item \textbf{Version control management:} Git
	\item \textbf{Development technologies:}
	
	\begin{itemize}
		\item HTML
		\item Java Persistence API (JPA)
		\item Java Persistence query language (JPQL)
		\item Asynchronous JavaScript and XML (AJAX)
	\end{itemize}
		
\end{enumerate}

\begin{flushleft}
			
\subsection*{System architecture} 
Java EE reference architecture used to archive scalability,security and reliability 

\subsection*{Frame work}
The system will be developed under Java EE frame works ,which includes JSF,JPA and JPQL, with addition of the Ajax framework for dynamic web pages on the client side.

\subsection*{Time Constraint} 
Given 2 weeks for the functional requirements, 1 week for the architectural requirements, 4 weeks for the implementation of the system and 1 week for testing.

\subsection*{Web services}
REST access channel will be used as is more lightweight and doesn't require a lot of bandwidth, this enhances the goal of one of the core quality requirements which is accessibility to many students, thus they can access via phone device web browsers with easy.

\subsection*{Environmental Constraints}
The system will be deployed at the University of Pretoria computer sciences server as an improvement to the current discussion board system.

\subsection*{Authentications}
LDAP will be used for login authentication as the system is constraint to the University of Pretoria Students and already the Department have been using LDAP.

\subsection*{Database Technology}
MySQL will be used as it enhances scalability , is open source ,thus will not cost university that much and have features for security that includes serialization, encrypting of passwords ,hashing and many more which can be implemented for strengthen security to the system.
		
\end{flushleft}
