%\documentclass{article}
 
%\begin{document}

	\begin{enumerate}
		\item Database technology: 
		\item Programming paradigm: 
		\item Programming language: 
		\item Development platform:  
		\item System architecture: 
		\item Architectural frameworks: 
		\item Architectural patterns: 
		\item Development technologies:
		\item IDE:NedBeans
		\item Build environment:
		\item Web server software:
		\item Web interface protocol:
		\item Client web browser:
		\item Client device operating systems:
		\item Source control management: 
		\item Text encoding: UTF-8
		\begin{itemize}
		
		
			\item{System architecture}
				Layered reference architecture will be used as it is enforced in the JAVA-EE framework
				
			\item{Time Constraint }		
					Given 2 weeks for the functional requirements, 1 week for the architectural requirements, 4 weeks 							for the implementation of the system and 1 week for testing.	
					
		
			\item{Web services}		
				REST access channel will be used as is more lightweight and doesn’t require a lot of bandwidth, this 						enhances the goal of one of the core quality requirements which is accessibility to many students, 					thus they can access via phone device web browsers with easy.
				
				\item{Environmental Constraints}
					The system will be deployed at the University of Pretoria computer sciences server as an 									improvement to the current discussion board system.
					
					\item{Authentications}
					LDAP will be used for login authentication as the system is constraint to the University of 									Pretoria Students and already the Department have been using LDAP.
					
					\item{Database Technology}
					MySQL will be used as it enhances scalability , is open source ,thus will not cost university that 					much and have features for security that includes serialization, encrypting of passwords ,hashing 						and many more which can be implemented for strengthen security to the system.
	\end{itemize}
		
					
				
				
	\end{enumerate}

%\end{document}
