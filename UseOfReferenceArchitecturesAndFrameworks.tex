\subsubsection{Reference Architecture}

The type of reference architecture that will be used is a Layered reference architecture and therefore 
Java-EE is the Layered reference architecture best suited for the Buzz system. Java-EE is the best choice
because it encompasses most of the quality requirements that we want to exploit which are scalability, security,
reliability and integrability.
\begin{itemize}
\item Apache CXF is a web service framework which could be used as it offers good performance with little computational overhead.It can be used with Java and Maven.
\item Apache Axis2 is essential for any web service.it is compatible with RESTful systems.It does not use a lot of memory and allows for deploying while the system is up and running.Allows the programmers to change and insert anything they wish.
\item Spring Framework is a application framework that can be used in conjunction with Java EE.It provides all the features we need such as working with a database,messaging and authentication.Objects can be managed through the use of the Inversion control container.
\item Spark is a web framework which will allow for easy routing of buzz.
\item Apache Tapestry is a view framework made up of components.it uses post and get when submitting the form and separates the html code from the Java code.
\end{itemize}