%\documentclass[a4paper,12pt]{report}
%\addtolength{\textwidth}{2cm}
%\addtolength{\topmargin}{-2cm}
%\addtolength{\textheight}{3.5cm}
%\newcommand{\HRule}{\rule{\linewidth}{0.5mm}}

%\begin{document}

\begin{itemize}
	\item \textbf{Scalability}
		There is a anticipated increase in entry volume. We are expecting an exponential increase in students using the system. It is predicted that this year 750 students are registered for COS132 by next year there will be close to a thousand. We expect a 25 percent growth rate over the next two years. 			\cite{classRepMeeting}
		The system must scale and incorporate with the standard Computer Science website. 
	\item \textbf{Performance Requirements}
		We will require a response time of no longer than a 1 sec. We expect about a maximum 100 searches of forum requests per second.With an average of about 5 requests per second. We however do not expect unindexed searches. To achieve both scalability and performance the following will be incoporated into the system.
		\begin{itemize}
			\item Real time access: combining system with Cloud computing.
			\item Data Partitioning: Spread data into multiple DB.
			\item Cache Engine (Dynamic Cache): instead of redo the same execution for same input parameters, we can remember the previous execution's result.
			\item Resources Pool: DBSession and TCP connection are expensive to create, so reuse them across multiple requests.
			\item Asynchronous Processing: continue with other processes, whilst waiting on the response of the other.
		\end{itemize}
	\item \textbf{Maintanability}
		The system will be comprise of modules and using a Singleton design pattern to handle communication. Thus a centralised node that all communication will pass through between modules. This will allow us the easily edit/add/remove modules from the system.
		\begin{itemize}
			\item Design for maintainability from the outset.
			\item Iterative development and regular reviews improve quality, e.g. Using the agile approach.
			\item Code readable that is easy to understand.
			\item Provide relevant documentation helps developers understand the software for further maintenance.
			\item Make use of automated builds make the code easy to compile.
			\item Make use of automated tests make it easy to validate changes.
			\item Application Architecture Standards: Multilayer design compliance.
			
		\end{itemize}
	\item \textbf{Reliability and Availability}
		Reliability will definitely be a priority. We will use a server that guarantees us at least 99 per uptime. This will probably be a University of pretoria server.
		\begin{itemize}
			\item Apply java OO and structured programming practices.
			\item Use good architectural infrastructure.
			\item Build management information into the application.
			\item Use redundancy for reliability, that is have multiple copies or services of the same type running at the same time.
			\item Use quality development tools, this includes HTML and the DOM, JavaScript debugging, profiling and auditing and so on.
			\item Use consistent error handling, and provide meaningful error messages.
			\item Elimination of single points of failure, by having multiple copies or services of the same type running at the same time, as specified before. So that when one service cannot be provided, then others or possibly the same services can be provided.
			\item Provide reliable crossover, enabling uptime to remain as high as possible even during maintenance.
			\item Detection of failures as they occur.
		\end{itemize}
		
	\item \textbf{Security}
		Clients must not have file access to the server and the root assess of the system must only be allowed to authorised personal. Designing strong password policies and using \textbf{security methods}.
		\begin{itemize}
			\item Input Validation
			\item Authentication
			\item Authorization
			\item Sensitive Data (Confidential information disclosure and data tampering)
			\item Session Management
			\item Cryptography
			\item Parameter Manipulation
			\item Exception Management
			\item Auditing and Logging
			\item Access control
			\item Auditing
		\end{itemize}
	\item \textbf{Monitorability and Auditability}
		All changes to the forums must be recorded and logged appropriately. This system must be fully audible. 
		\item \textbf{Usability}
		The web pages must be usable and follow a responsive design approach. This means the pages will display correctly on each screen size. We will follow google material design standards to ensure responsiveness.
		\begin{itemize}
			\item Look-and-feel: includes making navigation easy, useful interface cues, good color choice, for easy reading and scanning.
			\item Navigation: to tell the user where they are, and enable the user to go somewhere else.
			\item Interface design: inform users of the task the interface can be used to complete and provide feedback to let users now what has been done.
			\item Information architecture: organize or structure contact pleasant to read manner, and make use of short phrases as much as possible.
		\end{itemize}
	\item \textbf{Testability}
		Each function will be tested. But mathematical proofs for correctness is not necessary. Each part of the system must be tested. 
	\item \textbf{Integrability}
		The system must be able to integrate into an existing computer science website. We must meet their standards when it comes to obtaining user information. We will conform to all communication standards to the system. Also provide our own integration library which will gain access through out singleton node. Lastly the following will enable the system to be both a testable and integrable
		\begin{itemize}
			\item Isolate the Ugly Stuff: "ugly stuff" is any kind of code or infrastructure that is complicated or laborious or just plain inconvenient to get into a test harness, or that makes tests run very slowly.
			\item Using Fakes to Establish Boundary Conditions: For instance instead of doing the data directly delegate to some other services.
			\item Separate Deciding from Doing: an action and deciding to take an action treated as two separate responsibilities.
			\item Small Tests before Big Tests: Small test often point you direct to point of failure or error, where else Big test have a lot of factors to consider, which makes debugging hard, so more small test and leading to easy debugging for Big test.

		\end{itemize} 
\end{itemize}

%\end{document}