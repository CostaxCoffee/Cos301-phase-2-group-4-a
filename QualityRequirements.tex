%\documentclass[a4paper,12pt]{report}
%\addtolength{\textwidth}{2cm}
%\addtolength{\topmargin}{-2cm}
%\addtolength{\textheight}{3.5cm}
%\newcommand{\HRule}{\rule{\linewidth}{0.5mm}}

%\begin{document}

\begin{itemize}
	\item \textbf{Scalability}
		With huge volumes of students expected the system must be scalable. We will use a scalable storage solution combined with LB(load Balancers) to accommodate the extra traffic to the network. Also using stateless nodes to redirect query to the relevant modules of our system.
	\item \textbf{Performance Requirements}
		We will measure the response times using special tools to obtain the time from a request to a full page load. The servers performance shall be measured using the server logs. We cannot improve performance if we cannot measure it. 
	\item \textbf{Maintanability}
		The system will be comprise of modules and using a Singleton design pattern to handle communication. Thus a centralised node that all communication will pass through between modules. This will allow us the easily edit/add/remove modules from the system.
		\begin{itemize}
			\item \textbf{Documentation}
				Using UML to keep source code organised. Automatic generation of classes.
			\item \textbf{Application Architecture Standards}
				Multilayer design compliance.
		\end{itemize}
	\item \textbf{Reliability and Availability}
		Reliability will definitely be a priority. We will use a server that guarantees us at least 99 per uptime. This will probably be a University of pretoria server. Apply java OO and structured programming practices.
	\item \textbf{Security}
		We will use the HTTPS protocol to ensure that all traffic is encrypted. Also a user authentication will play a vital roll to ensure that there is no unauthorised users. The firewall must only allow the relevant ports to pass through to our server. 
	\item \textbf{Monitorability and Auditability}
		The server logs will allow us to monitor the system. These will be provided to us by the systems reporting module. Linux server logs will also allow us to audit the system and monitor its state.
	\item \textbf{Testability}
		Each function will be tested. But mathematical proofs for correctness is not necessary. Each part of the system must be tested. 
	\item \textbf{Usability}
		The web pages must be usable and follow a responsive design approach. This means the pages will display correctly on each screen size. We will follow google material design standards to ensure responsiveness.
	\item \textbf{Integrability}
		The system must be able to integrate into an existing computer science website. We must meet their standards when it comes to obtaining user information. We will conform to all communication standards to the system. Also provide our own integration library which will gain access through out singleton node.
\end{itemize}

%\end{document}