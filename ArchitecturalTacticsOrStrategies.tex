%\documentclass[a4paper,12pt]{report}
%\addtolength{\textwidth}{2cm}
%\addtolength{\topmargin}{-2cm}
%\addtolength{\textheight}{3.5cm}
%\newcommand{\HRule}{\rule{\linewidth}{0.5mm}}

%\begin{document}

Performance is one of the most important quality requirements that must be fully incorporated into the system.  This includes real time access; student will be reluctant to use a system with frequent delays. Asynchronous processing will increase the systems performance, so will data partitioning, resources pooling and others specified above.  Performance is closely related to scalability, so these tactics will also enable scalability.\\
	
Maintenance becomes easy if you design for maintainability from the outset, and be agile.
Coding readable code that is easy to understand reduces documentation, of which is ideal, but unfortunately documentation is still required.\\
 	
Systems that are available, reliable and secure encourage frequent usage.
Student study or work all day, and mostly at night, so access will be available at all times. As much as a security is top priority, access to the system should be short and one-time sign-in’s, but in the background intense security measures can take place, e.g. input validation, authentication, and encryption.\\

It is also ideal to have small test, to test each individual component or functionality, so as to make it easy to trace errors, if none, the big tests can be done. Testing is for developers but usability is strictly about users, so the systems must be easy to use, easy to remember how to use, as such will encourage users to use the system again.\\

%\end{document}