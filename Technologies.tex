We have chose the following technologies carefully as they incorporate into one another to form a sound basis for buzz space to operate on.

\begin{itemize}
	\item \textbf{Java EE}
		As mentioned in previous sections, Java EE will be best suited for the layered reference architecture that we want to use in our system. Java EE ensures efficient separation of layers when required in developing a system. Different layers is however what we want to achieve in our system.
	\item \textbf{Apache Maven}
		A great build tool that will be used for software project management.
	\item \textbf{HTTPS}
		We will use this technology to help reach our security quality requirement. It will add encryption to the website as a security method.
	\item \textbf{Apache Server}
		This is the web server we have chosen for the project. It will integrate effortlessly into Maven. This will allow us to deploy our project quickly. It is also an industry standard which will add towards our Reliability and Availability quality requirement. 
	\item \textbf{Java Persistence API (JPA) }
		We will use this API to help with the database integration. Thus we will gain scalability as required.
	\item \textbf{mySQL}
		As mySQL fits nicely into the java persistence API we will us this to amplify the benefits of using the JPA.
		
		\item \textbf{Bootstrap}
		Used in the HTML and interface part of the system for a competitive and easy presentation across all browsers 	and mobile browsers .
		\item \textbf{SMTP} The standardized mail protocol used to send mail and this is vital, as one of the features of buzz is to send emails to the users.
		\item \textbf{IPSec} This is a security technology which secures IP addresses which will be useful to help secure buzz.
	\item \textbf{JavaScript/JQuery}
		Used in the client side of validation of registration and login details.
	\item \textbf{WSDL} 
		We will use this technology to determine the function requirements and operations that will be required.
	\item \textbf{CORBA}
		This API will be used in the integration process independent from other technology's implementation, location, networking 	technologies and protocols.
	\item \textbf{IDL}
		This is a programming language that will be used for data analysis. 
		
		\item \textbf{JSF} This is a Java specification for building component-based user interfaces for web applications.
			
			\item \textbf{JDBC} Defines how a client may access a database. It provides methods for querying and updating data in a database. JDBC is oriented towards relational databases.
			
			\item \textbf{EJB} An architecture for setting up program components, written in the Java programming language, that run in the server parts of a computer network.
			
			\item \textbf{JAX-RS} An API that provides support in creating web services according to the REST architectural pattern. JAX-RS uses annotations to simplify the development and deployment of web service clients and endpoints.
			
			\item \textbf{Glassfish}  Reference implementation of Java EE and as such supports EJB, JPA, JavaServer Faces, JMS, RMI, JavaServer Pages, servlets, etc. This allows developers to create enterprise applications that are portable and scalable and that integrate with legacy technologies. Optional components can also be installed for additional services.
			
			\item \textbf{JUnit Testing} A unit testing framework for the Java programming language. JUnit has been important in the development of test-driven development.
		
\end{itemize}