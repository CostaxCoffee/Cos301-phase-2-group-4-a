We have chose the following technologies carefully as they incorporate into one another to form a sound basis for buzz space to operate on.

\begin{itemize}
	\item \textbf{Java EE}
		This was a recommendation from the specifications given to us by the client.
	\item \textbf{Apache Marven}
		A great build too that will be used for software project management.
	\item \textbf{HTTPS}
		We will use this technology to help reach our security quality requirement. It will add encryption as a security method.
	\item \textbf{Apache Server}
		This is the web server we have chosen for the project. It will integrate effortlessly into Maven. This will allow us to deploy our project quickly. It is also an industry standard which will add towards out Reliability and Availability quality requirement. 
	\item \textbf{Java Persistence API (JPA) }
		We will use this API to help with the database integration. Thus we will gain scalability as required.
	\item \textbf{mySQL}
		As mySQL fits nicely into the java persistence API we will us this tos amplify the benefits of using the JPA.
		
		\item \textbf{Bootstrap}
		Used in the html and interface part of the system for a competitive and easy presentation across all browsers 	and mobile browsers .
		\item \textbf{SMTP} is the standardized mail protocol used to send mail and this is vital as one of the features of buzz is to send emails to the users.
		\item \textbf{IPSec} is a security technologies which secures IP addresses which will be useful to help secure buzz.
	\item \textbf{JavaScript/JQuery}
		Used in the client side of validation of registration and login details
		
			\item \textbf{JSF:}
			\item \textbf{JDBC:}
			\item \textbf{EJB:}
			\item \textbf{JAX-RS:}
			\item \textbf{Glassfish:}
			\item \textbf{JUnit Testing:}
		
\end{itemize}